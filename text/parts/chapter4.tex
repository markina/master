\chapter{Описание гибридного алгоритма и эмпирический анализ времени его работы}
\label{chapter4}

В данной главе описан гибридный алгоритм, приведен асимптотический анализ времени его работы, описана реализация гибридного алгоритма, а также приведен эмпирический анализ времени работы реализованного алгоритма в сравнении с уже существующими.

\section{Гибридный алгоритм}

\subsection{Формулировка гибридного алгоритма}

Теперь мы можем сформулировать гибридный алгоритм. Мы используем алгоритм Буздалова как базовый алгоритм. Каждый раз прежде чем вызывать процедуры $HelperA$ и $HelperB$, мы проверяем размер точек, на которых необходимо выполнить сортировку. Если точки с таким размером эффективно сортируются в алгоритме ENS-NDT-ONE, гибридный алгоритм переключается с базового алгоритма на алгоритм ENS-NDT-ONE для решения этой подзадачи. Так как алгоритм ENS-NDT-ONE приспособлен к гибридизации, в том числе невосприимчив к отсутствию монотонности, и учитывает предпоставленные ранги в базовом алгоритме, полученный алгоритм будет работать корректно.

Если говорить более формально, мы определяем для каждого значения размерности пороговое значение, которое означает, что каждая подзадача с таким и меньшим размером точек должна быть делегирована алгоритму ENS-NDT-ONE. Для процедуры $HelperA$ размер множества точек {---} это размер множества $S$, для процедуры $HelperB$ {---} это сумма размеров множеств $L$ и $R$.

\subsection{Анализ времени работы гибридного алгоритма}

В этом разделе дадим некоторую оценку асимптотики времени работы гибридного алгоритма.

Напомним, что время работы алгоритма Буздалова асимптотически равно $O(N(\log N)^{M-1})$. Так как мы определяем пороговые значения как константы, асимптотическая оценка времени работы итогового гибридного  алгоритма по-прежнему равна $O(N(\log N)^{M-1})$. Заметим, что более тщательный выбор пороговых значений, например, учитывающий особенности входных данных, может улучшить асимптотическую оценку. Так как эта задача является сложной, мы оставляем это для возможной будущей работы.

\section{Реализация гибридного алгоритма}

В данном разделе будут описаны подробности реализации гибридного алгоритма.

Так как при создании алгоритма ENS-NDT-ONE, мы учитывали, что точки могут иметь предпоставленные ранги, адаптировать алгоритм для использования его в гибридном алгоритме не составляет труда.

\subsection{HelperA}

При переключении на алгоритм ENS-NDT-ONE с алгоритма Буздалова при вызове процедуры $HeplerA$ требуется дополнительно учитывать только то, что точки имеют изначальный ранг. Таким образом, помимо множества точек в функцию приходят ранги точек. Обновление ранга рассматриваемой точки происходит, только если изначальный ранг был меньше нового.

% TODO привести ENS-NDT-ONE-HelperА
% TODO привести моменты переключения с алгоритма Буздалова

\subsection{HelperВ}

Переключения на алгоритм ENS-NDT-ONE при вызове процедуры $HeplerB$ значительно сложнее. Псевдокод процедуры $HeplerB$ гибридного алгоритма на основе алгоритма ENS-NDT-ONE представлен на Листинге~\ref{procedure_end_ndt_one_helper_b}. Множество точек $L$ уже окончательно проранжированы в базовом алгоритме Буздалова, множество точек $R$ имеют некоторые предварительно предпоставленные ранги. Задача метода обновить ранги точек множества $R$ на основе рангов точек множества $L$. 

\begin{algorithm}
\begin{algorithmic}[1]
\Procedure{ENS-NDT-ONE-HelperB}{$L, R, M, B, Ranks$}
    \State{$S \gets CreateSplits(L, M-1,B)$}
    \State{$\mathcal{T} \gets new NDTreeOne(S, B)$}
    \State{$InsertIntoNDTreeOne(\mathcal{T}, P_1)$}
    \State{$j \gets 1$}
    \For{$p \in L \cup R$}
        \State {$r \gets p.rank$}
        \If{$p \in L$} 
            \State{$InsertIntoNDTreeOne(\mathcal{T}, p, r)$}
        \EndIf
        \If{$p \in R$} 
            \State{$r \gets FindRankInNDTreeOne(\mathcal{T}, p, r)$}
        \EndIf
    \EndFor
    \State{\Return {$\mathcal{Ranks}$}}
\EndProcedure
\end{algorithmic}
\caption{Главная процедура алгоритма ENS-NDT-ONE, адаптированная для переключения в момент $HeplerB$.}
\label{procedure_end_ndt_one_helper_b}
\end{algorithm}

Первым интересным моментом является то, что $split$ структуру мы будем строить только для множества точек $L$, то есть точки из множества $R$ никак не влияют друг на друга и обновляются только на основе рангов точек $L$. Также добавлять в структуру мы будем только точки из множества $L$, а точки из $R$ мы будем только ранжировать. Так как точки приходят из базового алгоритма в отсортированном порядке дополнительно делать лексикографическую сортировку нет необходимости. Таким образом мы перебираем объединение точек $L$ и $R$ в лексикографическом порядке. Далее в зависимости от вида точки либо обновляем ей ранг, либо добавляем точку в структуру для последующего ранжирования других точек.

Таким образом, мы реализовали гибридный алгоритм на основе алгоритма Буздалова и алгоритма Густавссона. 

\section{Настройка параметров гибридного алгоритма}

Настройка гибридного алгоритма будет представлять некоторый диапазон размеров множеств точек, при котором происходит переключение, зависящий от размерности. Параметры основаны на экспериментальных данных и не зависят от размера множеств точек на которых изначально запускается гибридный алгоритм недоминирующей сортировки. Другими словами, эти параметры можно считать константами. 

Наше экспериментальное исследование показало, что для размерности три оптимальным будет переключение, когда размер множества точек не превышает $100$, а при размерностях больше трех переключение необходимо осуществлять на множествах точек размером не более $20 000$. 

\section{Сравнение с существующими алгоритмами на искусственно сгенерированных тестовых данных}

В данном разделе приводится сравнение эффективности работы нового гибридного алгоритма с существующими алгоритмами недоминирующей сортировки. Сравнение производилось с двумя родительскими алгоритмами, которые в свою очередь являются лучшими алгоритмами недоминирующей сортировки на сегодняшний день, и с алгоритмом ENS-NDT-ONE, работающим самостоятельно.

Замеры времени работы производились на множестве точек размером до $10^6$ с размерностями $3, 5, 7, 10$ и $15$ на случайно сгенерированных независимых точках в гиперкубе $[0; 1]^M$ и на точках расположенных, на одной гиперплоскости и имеющих один ранг. Результаты приведены в Таблице~\ref{results}, серым обозначен лучший в каждой группе алгоритм.

\newcommand{\best}{\cellcolor{gray!50!white}}

\begin{table}[!ht]
\caption{Среднее время работы алгоритмов в секундах. Лучшее время в каждой категории обозначено серым цветом.}\label{results}
\begin{tabular}{rr|rr|rr|rr|rr}
$N$&$M$ & \multicolumn{2}{c|}{Divide\&Conquer} 
        & \multicolumn{2}{c|}{ENS-NDT} 
        & \multicolumn{2}{c|}{ENS-NDT-ONE} 
        & \multicolumn{2}{c}{Hybrid} \\
& & {\scriptsize hypercube} & {\scriptsize hyperplane} 
  & {\scriptsize hypercube} & {\scriptsize hyperplane} 
  & {\scriptsize hypercube} & {\scriptsize hyperplane} 
  & {\scriptsize hypercube} & {\scriptsize hyperplane} \\\hline
$5\cdot10^5$&$3$  & $1.52$ & $0.85$ & $1.95$ & $0.73$ & $1.66$ & $0.76$ & \best $1.17$ & \best $0.67$\\
      $10^6$&$3$  & $2.82$ & $1.60$ & $5.25$ & $1.61$ & $4.25$ & $1.65$ & \best $2.63$ & \best $1.50$\\\hline
$5\cdot10^5$&$5$  & $22.7$ & $16.6$ & $8.31$ & \best $2.01$ & \best $6.25$ & $2.22$ & $6.43$ & $4.68$\\
      $10^6$&$5$  & $45.2$ & $33.0$ & $26.3$ & \best $5.22$ & $18.2$ & $5.82$ & \best $17.2$ & $12.8$\\\hline
$5\cdot10^5$&$7$  & $89.6$ & $55.1$ & $17.1$ & $6.96$ & $15.5$ & \best $6.78$ & \best $9.29$ & $7.02$\\
      $10^6$&$7$  & $191.5$& $120.2$& $55.4$ & $19.4$ & $46.1$ & \best $18.9$ & \best $26.8$ & $20.1$\\\hline
$5\cdot10^5$&$10$ & $197.7$& $99.9$ & $27.6$ & $15.9$ & $36.7$ & $17.7$ & \best $14.5$ & \best $11.5$\\
      $10^6$&$10$ & $478.8$& $228.6$& $84.8$ & $48.1$ & $104.8$& $55.0$ & \best $41.0$ & \best $33.0$\\\hline
$5\cdot10^5$&$15$ & $190.0$& $116.1$& $40.8$ & $23.0$ & $62.1$ & $25.9$ & \best $22.6$ & \best $15.7$\\
      $10^6$&$15$ & $587.9$& $337.5$& $135.4$& $76.3$ & $206.8$& $85.4$ & \best $64.5$ & \best $46.0$\\\hline
\end{tabular}
\end{table}

Для каждой конфигурации ввода было создано $10$ экземпляров входных данных. Мы измерили общее время во всех этих случаях и разделили их на $10$, чтобы получить среднее время выполнения. Измерения времени выполнялись с использованием пакета Java Microbenchmark Harness с одной итерацией прогрева не менее 6 секунд, чего было достаточно для стабилизации работы программы. Был использован высокопроизводительный сервер с процессорами AMD OpteronTM 6380 и 512 GB ОЗУ, а код был запущен с виртуальной машиной OpenJDK 1.8.0 141.
Репозиторий с кодом представлен на GitHub~\footnote{https://github.com/mbuzdalov/non-dominated-sorting/releases/tag/v0.1}. Репозиторий также содержит графики времени работы. В Таблице~\ref{results} показаны только средние результаты для $N = 5 \cdot 10^5$ и $10^6$. Видно, что гибридный алгоритм выигрывает во всех случаях, кроме $M = 5$ и $M = 7$ на гиперплоскости. Еще одно интересное наблюдение, что ENS-NDT-ONE работает быстрее, чем ENS-NDT, на экземплярах гиперкуба с $M \leq 7$, что означает, что предложенная эвристика действительно эффективна.

%TODO добавить графики из репозитория

\section{Сравнение времени работы алгоритмов на худшем случае}

Так как время работы алгоритма Густавссона ENS-NDT и нового алгоритма ENS-NDT-ONE имеют квадратичную асимптотику $O(MN^2)$ в некотором худшем случаем, мы провели сравнение времени работы этих алгоритмов на таком виде данных с получившимся гибридным алгоритмом. 

Мы провели исследование на множестве точек размером до $10^5$ и разных размерностях: $3, 5, 7, 10$ и $15$. Результаты сравнения времени работы на худшем случае для алгоритмов ENS-NDT и ENS-NDT-ONE приведены в Таблице~\ref{results_worst_case}.

\begin{table}[!ht]
\caption{Среднее время работы алгоритмов на худшем виде данных для алгоритмов ENS-NDT и ENS-NDT-ONE в секундах.}
\label{results_worst_case}
\begin{tabular}{rr|r|r|r|r}
$N$&$M$ & {Divide\&Conquer} 
        & {ENS-NDT} 
        & {ENS-NDT-ONE} 
        & {Hybrid} \\
& & {\scriptsize worst case} 
  & {\scriptsize worst case} 
  & {\scriptsize worst case} 
  & {\scriptsize worst case} \\\hline
      $10^5$&$3$  & $0.04$& $0.32$ & $0.38$ & $0.04$\\\hline
      $10^5$&$5$  & $0.05$& $4.6$ & $7.5$ & $0.7$\\\hline
      $10^5$&$7$  & $0.05$& $45.5$ & $38.3$ & $1.58$\\\hline
      $10^5$&$10$ & $0.06$& $69.1$ & $81.8$ & $3.37$\\\hline
      $10^5$&$15$ & $0.08$& $175.3$ & $185.3$ & $4.45$\\\hline
\end{tabular}
\end{table}

Эмпирический анализ времени работы показал, что данные, которые только за квадратичное время сортируются алгоритмами ENS-NDT и ENS-NDT-ONE, нашим гибридным алгоритмом сортируются несравнимо быстро. Несмотря на то, что гибридный алгоритм проигрывает по времени алгоритму на основе метода <<разделяй и властвуй>>, это происходит только на специфическом, худшем виде данных, на остальных же наборах точек, как было видно в Таблице~\ref{results_worst_case}, гибридный алгоритм работает сильно быстрее алгоритма Буздалова. Это означает, что гибридный алгоритм обладает большей универсальностью по сравнению с другими алгоритмами, так он на всех наборах входных данных либо является самым эффективным, либо незначительно уступает самому эффективному алгоритму.

