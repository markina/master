\startprefacepage

Многие задачи оптимизации в реальном мире являются многокритериальными, то есть требуется максимизировать или минимизировать критерии объектов, которые часто противоречат друг другу. Выбрать единственное решение обычно невозможно, вместо этого получают множества несравнимых объектов. Поиск разнообразных несравнимых решений часто используется в многокритериальных эволюционных алгоритмах.

В области эволюционных алгоритмов рассматривают три основных подхода \cite{Brockhoff}: на основе Парето фронтов, на основе индикаторов и на основе декомпозиции. Хотя существуют хорошо известные алгоритмы на основе декомпозиции \cite{Zhang2007} и на основе индикаторов \cite{Zitzler}, большинство современных алгоритмов основано на Парето: PESA-II \cite{PESA-II}, NSGA-II \cite{NSGA-II}, \cite{Deb2013}, SPEA2 \cite{SPEA2} и многие другие. Алгоритмы, основанные на Парето, делятся на следующие группы в зависимости от того, как выбираются или ранжируются решения: алгоритмы, поддерживающие недопустимые решения \cite{Coello, PESA-II, PAES}, алгоритмы, основанные на недоминирующей сортировке \cite{NSGA-II, Deb2013}, алгоритмы, использующие показатель доминирования \cite{Fonseca} или силу доминирования \cite{SPEA2}. В данной дипломной работе сфокусировано внимание на недоминирующей сортировке, так как многие популярные эволюционные алгоритмы ее используют \cite{NSGA-II, Deb2013}.

Цель данной работы -- разработать алгоритм недоминирующей сортировки, который будет подходить для крупномасштабной многокритериальной оптимизации. Итоговый алгоритм будет гибридом двух хорошо известных алгоритмов: на основе метода ``разделяй и властвуй'', изначально предложенного Йенсеном \cite{Jensen}, и на основе недоминирующего дерева, предложенного Густавссоном и Соберфильдтом \cite{Gustavsson}.

Предложенный алгоритм во-первых в худшем случае имеет асимптотику алгоритма на основе метода ``разделяй и властвуй'', во-вторых он эффективен на практике. Наше экспериментальное исследование показало ускорение обоих родительских алгоритмов на основных видах данных: на равномерно распределенных точках в гиперкубе и на точках в одной гиперплоскости, имеющих один фронт. Размеры множеств точек достигали $10^6$ и с размерностью до $15$.

В Главе 1 данной магистерской работы представлен общий обзор. 
В разделе 1.1 приведены основные понятие, определение недоминирующей сортировки, а также для подтверждение актуальности данной работы, представлены примеры ее применения. 
В разделе 1.2 представлено краткий список существующих алгоритмов недоминирующей сортировки и рассмотрены подробно кандидаты для создания гибрида. 
В разделе 1.3 описаны слабые места алгоритмов и условия при которых возможно создание нового гибридного алгоритма. 
В разделе 1.4 сформулирована итоговая задачи магистерской диссертации.

В Главе 2 представлены теоретические исследования существующих алгоритмов и приведено теоретическое обоснование разработанного алгоритма недоминирующей сортировки с асимптотической оценкой времени работы.
В разделе 2.1 приведено сравнение существующих алгоритмов. 
В разделе 2.2 описана предлагаемая схема гибридного алгоритма 
В разделе 2.3 описана попытка адаптации существующих алгоритмов для создания гибрида. 
В разделе 2.4 представлен новый алгоритм недоминирующей сортировки с доказательством асимптотики времени работы, который впоследствии используется в гибридном алгоритме.
В разделе 2.5 представлена итоговая формулировка гибридного алгоритма и приведено доказательство асимптотики времени работы гибридного алгоритма.

В Главе 3 представлена реализация нового и гибридного алгоритма и проведен эмпирический анализ времени работы новых алгоритмов в сравнении с существующими.
В разделе 3.1 представлены детали реализации нового алгоритма ENS-NDT-ONE
, основанного на алгоритме Густавссона и Соберфильдта. 
В разделе 3.2 описана реализация гибридного алгоритма на основе алгоритма Буздалова и Густавссона.
В разделе 3.3 описан подход, который мы использовали для настройки получившегося алгоритма.
В разделе 3.4 представлены результаты сравнения времени работы получившегося алгоритма с существующими на искусственных данных.
В разделе 3.4 представлены результаты сравнения времени работы алгоритмов на худшем случае. 
В разделе 3.6 представлена многопоточная версия гибридного алгоритма.

В заключении представлены итоги работы и предложены некоторые идеи дальнейшего пути развития гибридных алгоритмов недоминирующей сортировки.