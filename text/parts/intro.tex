\startprefacepage

Многие задачи оптимизации в реальном мире являются многокритериальными, то есть требуется максимизировать или минимизировать критерии объектов, которые часто противоречат друг другу. Выбрать единственное лучшее решение обычно невозможно, вместо этого получают множества несравнимых объектов. Поиск разнообразных несравнимых решений часто используется в многокритериальных эволюционных алгоритмах.

В области эволюционных алгоритмов рассматривают три основных подхода~\cite{Brockhoff}: на основе Парето фронтов, на основе индикаторов и на основе декомпозиции. Хотя существуют хорошо известные алгоритмы на основе декомпозиции~\cite{Zhang2007} и на основе индикаторов~\cite{Zitzler}, большинство современных алгоритмов основано на Парето фронтах: PESA-II~\cite{PESA-II}, NSGA-II~\cite{NSGA-II},~\cite{Deb2013}, SPEA2~\cite{SPEA2} и многие другие. Алгоритмы, основанные на Парето фронтах, в свою очередь, делятся на следующие группы в зависимости от того, как выбираются или ранжируются решения: алгоритмы, поддерживающие недопустимые решения~\cite{Coello, PESA-II, PAES}, алгоритмы, основанные на недоминирующей сортировке~\cite{NSGA-II, Deb2013}, алгоритмы, использующие показатель доминирования~\cite{Fonseca} или силу доминирования~\cite{SPEA2}. В данной дипломной работе сфокусировано внимание на недоминирующей сортировке, так как многие популярные эволюционные алгоритмы ее используют~\cite{NSGA-II, Deb2013}.

Цель данной работы {---} разработать алгоритм недоминирующей сортировки, который будет подходить для крупномасштабной многокритериальной оптимизации. Итоговый алгоритм является гибридом двух хорошо известных алгоритмов: алгоритма на основе метода <<разделяй и властвуй>>, изначально предложенного Йенсеном~\cite{Jensen}, и нового алгоритма на основе недоминирующего дерева, который был предложен в данной работе.

Предложенный алгоритм, во-первых, в худшем случае имеет асимптотику алгоритма на основе метода <<разделяй и властвуй>>, во-вторых, он эффективен на практике. Наше экспериментальное исследование показало ускорение по сравнению с обоими родительскими алгоритмами на основных видах данных: на равномерно распределенных точках в гиперкубе и на точках в одной гиперплоскости, имеющих один фронт. Размеры множеств точек достигали $10^6$, а их размерность {---} $15$.

В Главе 1 данной магистерской работы представлен обзор существующих алгоритмов и введены необходимые для понимания работы понятия. 
В Разделе 1.1 приведены основные понятия, определение недоминирующей сортировки, а также для подтверждения актуальности данной работы, представлены примеры ее применения. 
В Разделе 1.2 представлен краткий список существующих алгоритмов недоминирующей сортировки и рассмотрены подробно кандидаты для создания гибрида. 
В Разделе 1.3 описаны недостатки существующих алгоритмов. 
В Разделе 1.4 сформулирована итоговая задача магистерской диссертации.

В Главе 2 представлены методы разработки гибридных алгоритмов недоминирующей сортировки.
В Разделе 2.1 приведен отбор алгоритмов, подходящих для создания гибридного алгоритма.
В Разделе 2.2 описана предлагаемая схема гибридного алгоритма.
В Разделе 2.3 описана попытка адаптации существующих алгоритмов для создания гибрида. 

В Главе 3 описан новый алгоритм недоминирующей сортировки ENS-NDT-ONE.
В Разделе 3.1 описан сам алгоритм. 
В Разделе 3.2 приведено доказательство асимптотической оценки времени работы.
В Разделе 3.3 представлены детали реализации нового алгоритма ENS-NDT-ONE.

В Главе 4 описан гибридный алгоритм и произведен эмпирический анализ времени его работы.
В Разделе 4.1 описан гибридный алгоритм, разработанный в данной работе.
В Разделе 4.2 приведены детали реализации.
В Разделе 4.3 описана настройка параметров гибридного алгоритма, обеспечивающая эффективную работу. 
В Разделе 4.4 произведено сравнение с существующими алгоритмами.
В Разделе 4.5 произведено сравнение с существующими алгоритмами на худшем случае.

В Главе 5 представлена многопоточная версия алгоритма недоминирующей сортировки.

В заключении представлены итоги работы и предложены некоторые идеи дальнейшего пути развития гибридных алгоритмов недоминирующей сортировки.