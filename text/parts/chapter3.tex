\chapter{Реализация разработанного алгоритма и эмпирический анализ времени работы}
\label{chapter3}

В данной главе рассмотрена реализация нового алгоритма ENS-NDT-ONE и детали реализации гибридного алгоритма, а также результаты эмпирического анализа времени работы реализованного алгоритма в сравнении с уже существующими алгоритмами.

\section{Реализация алгоритма ENS-NDT-ONE}

В данном разделе будет описана реализация алгоритма ENS-NDT-ONE. 

Вдохновившись идеями алгоритма ENS-NDT, был создан алгоритм ENS-NDT-ONE. Основным изменением алгоритма ENS-NDT стало то, что вместо множества деревьев для каждого ранга, теперь в структуре одно дерево. На листинге ~\ref{procedure_end_ndt_one} приведен псевдокод основного метода этого алгоритма, который принимает в качестве аргументов множество точек $P$, $M$ {---} размерность и $B$ {---} порог, максимальное количество точек в вершине. Для получения $split$ структуры используется функция $CreateSplits$, о которой можно почитать в первой главе данной работы.

Помимо этого, для ускорения процесса определения ранга, будем хранить максимальный ранг на поддереве, что позволит добавить следующую эвристику: если у поддерева максимальный ранг меньше ранга рассматриваемой точки, то это поддерево никак не может повлиять на ранг данной точки. 

\begin{algorithm}
\begin{algorithmic}[1]
\Procedure{ENS-NDT-ONE}{P, M, B}
    \State{$P \gets Sort(P, a^M \prec b^M, ..., a^1 \prec b^1)$}
    \State{$S \gets CreateSplits(P, M-1,B)$}
    \State{$\mathcal{F} \gets \{\{P_1\}\}$}
    \State{$\mathcal{T} \gets new NDTreeOne(S, B)$}
    \State{$InsertIntoNDTreeOne(\mathcal{T}, P_1)$}
    \State{$j \gets 1$}
    \For{$i = 2, ..., |P|$}
        \If{$P_{i-1} \neq P_{i}$}
            \State {$j \gets FindRankInNDTreeOne(\mathcal{T}, P_i)$}
            \State{$InsertIntoNDTreeOne(\mathcal{T}, \mathcal{P}_i)$}
        \EndIf
        \State{$\mathcal{F}_j \gets \mathcal{F}_j \cup {P_i}$}
    \EndFor
    \State{\Return {$\mathcal{F}$}}
\EndProcedure
\end{algorithmic}
\caption{Главная процедура алгоритма ENS-NDT-ONE.}
\label{procedure_end_ndt_one}
\end{algorithm}

В функцию $FindRankInNDTreeOne$ будет добавлен дополнительный аргумент, ранг точки $P_i$, тогда функция $FindRankInNDTreeOne$ в терминальной вершине будет иметь реализацию представленную на листинге ~\ref{procedure_find_rank_term}. А в нетерминальной вершине реализация представляет из себя два рекурсивных вызова на вершинах-потомках с одним только отсечением, если координата рассматриваемой вершины больше либо равна медианному значению, то в левого ребенка можно не заходить, то есть в поддерево, где по текущей координате все точки больше рассматриваемой, не найдется ни одной точки, которая бы доминировала нашу, следовательно заходить в такое поддерево нет смысла. На листинге ~\ref{procedure_find_rank_n_term} представлен псевдокод определения ранга в нетерминальной вершине.

\begin{algorithm}
\begin{algorithmic}[1]
\Procedure{FindRankInNDTreeOne}{$\mathcal{T}, p, r$}
    \If {$ maxRank < r$}
        \State {\Return {r}}
    \EndIf

    \If{$p[\mathcal{T}.splitCoordinate] >= \mathcal{T}.splitValue$}
        \State {$r \gets FindRankInNDTreeOne(\mathcal{T}.worseNode, P_i, r)$}     
    \EndIf
    \State {$r \gets FindRankInNDTreeOne(\mathcal{T}.betterNode, P_i, r)$}     
    
    \State{\Return {$r$}}
\EndProcedure
\end{algorithmic}
\caption{Процедура поиска ранга точки с предварительным рангом в нетерминальной вершине.}
\label{procedure_find_rank_n_term}
\end{algorithm}

\begin{algorithm}
\begin{algorithmic}[1]
\Procedure{FindRankInNDTreeOne}{$\mathcal{T}, p, r$}
    \If {$maxRank < r$}
        \State {\Return {r}}
    \EndIf
    \For{$i = |\mathcal{T}.points|, ..., 1$}
        \If {$\mathcal{T}.points[i] \prec p$}
            \State {\Return {$\mathcal{T}.ranks[i] + 1$}}
        \EndIf
    \EndFor
    \State{\Return {$r$}}
\EndProcedure
\end{algorithmic}
\caption{Процедура поиска ранга точки с предварительным рангом в терминальной вершине.}
\label{procedure_find_rank_term}
\end{algorithm}

Важно отметить, что если ранг на поддереве $< r$, то есть ранга рассматриваемой точки, то текущее поддерево не может повлиять на ранг точки, для которой происходит обновления ранга. В таком случае в строке 3 листингов ~\ref{procedure_find_rank_term} и ~\ref{procedure_find_rank_n_term} происходит выход из процедур.

\section{Реализация гибридного алгоритма}

В данном разделе будут описаны подробности реализации гибридного алгоритма.

Так как при создании алгоритма ENS-NDT-ONE, мы учитывали, что точки могут иметь предпоставленные ранги, адаптировать алгоритм для гибридизации не составляет труда 

\subsection{HelperA}

Адаптация для функции $HeplerA$ полностью совпадает с самим алгоритмом ENS-NDT-ONE, учитывая только то, что точки имеют на изначальный ранг. То есть помимо множества точек в функцию приходят ранги точек. Обновление ранга рассматриваемой точки происходит, только если изначальный ранг был меньше нового.

\subsection{HelperВ}

Для функции $HeplerB$ немного сложнее, представим псевдокод основного метода алгоритма ENS-NDT-ONE на листинге ~\ref{procedure_end_ndt_one_helper_b}. Множество точек $L$ уже окончательно проранжированы в базовом алгоритма ``Разделяй и властвуй'', множество точек $R$ имеют некоторые предварительно предпоставленные ранги. Задача метода обновить ранги точек множества $R$ на основе рангов точек множества $L$. 

\begin{algorithm}
\begin{algorithmic}[1]
\Procedure{ENS-NDT-ONE-HelperB}{$L, R, M, B, Ranks$}
    \State{$S \gets CreateSplits(L, M-1,B)$}
    \State{$\mathcal{T} \gets new NDTreeOne(S, B)$}
    \State{$InsertIntoNDTreeOne(\mathcal{T}, P_1)$}
    \State{$j \gets 1$}
    \For{$p \in L \cup R$}
        \State {$r \gets p.rank$}
        \If{$p \in L$} 
            \State{$InsertIntoNDTreeOne(\mathcal{T}, p, r)$}
        \EndIf
        \If{$p \in R$} 
            \State{$r \gets FindRankInNDTreeOne(\mathcal{T}, p, r)$}
        \EndIf
    \EndFor
    \State{\Return {$\mathcal{Ranks}$}}
\EndProcedure
\end{algorithmic}
\caption{Главная процедура алгоритма ENS-NDT-ONE, адаптированная для переключения в момент $HeplerB$.}
\label{procedure_end_ndt_one_helper_b}
\end{algorithm}

Первым интересным моментом является то, что $split$ структуру мы будем строить только для множества точек $L$, то есть точки из множества $R$ никак не влияют друг на друга и обновляются только на основе рангов точек $L$. Так же добавлять в структуру мы будет только точки из множества $L$, а точки из $R$ мы будем только ранжировать. Так как точки приходят из базового алгоритма в отсортированном порядке дополнительно делать лексикографическую сортировку нет необходимости. Таким образом мы перебираем объединение точек $L$ и $R$ в лексикографическом порядке. Далее в зависимости от вида точки либо обновляем ей ранг, либо добавляем точку в структуру для последующего ранжирования других точек.

Таким образом, мы реализовали гибридный алгоритм на основе алгоритма Буздалова и алгоритма Густавссона. 

\section{Настройка параметров гибридного алгоритма}

Настройка гибридного алгоритма будет представлять некоторый диапазон размеров множеств точек для каждой размерности, при котором происходит переключение. Параметры основаны на экспериментальных данных и не зависят от размера множеств точек на которых изначально запускается гибридный алгоритм недоминирующей сортировки. Другими словами, эти параметры можно считать константами. 

Наше экспериментальное исследование показало, что для размерности три оптимальным переключением будет, когда размер множества точек не превышает $100$, а при размерностях больше трех переключение необходимо осуществлять на множествах точек размером не более $20 000$. 

\section{Сравнение с существующими алгоритмами на искусственно сгенерированных тестовых данных}

В данном разделе приводится сравнение эффективности работы нового гибридного алгоритма с существующими. Сравнение производилось с двумя родительскими алгоритмами, которые в свою очередь являются лучшими алгоритмами недоминирующей сортировки на сегодняшний день, и с алгоритмом END-NDT-ONE работающим самостоятельно.

Замеры времени работы производились на множестве точек размером до $10^6$ с размерностями 3, 5, 7, 10, 15, на случайно сгенерированных независимых точках в гиперкубе $[0; 1]^M$ и на точках расположенных на одной гиперплоскости и имеющих один ранг. Результаты приведены в таблице ~\ref{results}, серым обозначены лучшие в каждой группе алгоритм.

\newcommand{\best}{\cellcolor{gray!50!white}}

\begin{table}[!ht]
\caption{Среднее время работы алгоритмов в секундах. Лучшее время в каждой категории обозначено серым цветом.}\label{results}
\begin{tabular}{rr|rr|rr|rr|rr}
$N$&$M$ & \multicolumn{2}{c|}{Divide\&Conquer} 
        & \multicolumn{2}{c|}{ENS-NDT} 
        & \multicolumn{2}{c|}{ENS-NDT-ONE} 
        & \multicolumn{2}{c}{Hybrid} \\
& & {\scriptsize hypercube} & {\scriptsize hyperplane} 
  & {\scriptsize hypercube} & {\scriptsize hyperplane} 
  & {\scriptsize hypercube} & {\scriptsize hyperplane} 
  & {\scriptsize hypercube} & {\scriptsize hyperplane} \\\hline
$5\cdot10^5$&$3$  & $1.52$ & $0.85$ & $1.95$ & $0.73$ & $1.66$ & $0.76$ & \best $1.17$ & \best $0.67$\\
      $10^6$&$3$  & $2.82$ & $1.60$ & $5.25$ & $1.61$ & $4.25$ & $1.65$ & \best $2.63$ & \best $1.50$\\\hline
$5\cdot10^5$&$5$  & $22.7$ & $16.6$ & $8.31$ & \best $2.01$ & \best $6.25$ & $2.22$ & $6.43$ & $4.68$\\
      $10^6$&$5$  & $45.2$ & $33.0$ & $26.3$ & \best $5.22$ & $18.2$ & $5.82$ & \best $17.2$ & $12.8$\\\hline
$5\cdot10^5$&$7$  & $89.6$ & $55.1$ & $17.1$ & $6.96$ & $15.5$ & \best $6.78$ & \best $9.29$ & $7.02$\\
      $10^6$&$7$  & $191.5$& $120.2$& $55.4$ & $19.4$ & $46.1$ & \best $18.9$ & \best $26.8$ & $20.1$\\\hline
$5\cdot10^5$&$10$ & $197.7$& $99.9$ & $27.6$ & $15.9$ & $36.7$ & $17.7$ & \best $14.5$ & \best $11.5$\\
      $10^6$&$10$ & $478.8$& $228.6$& $84.8$ & $48.1$ & $104.8$& $55.0$ & \best $41.0$ & \best $33.0$\\\hline
$5\cdot10^5$&$15$ & $190.0$& $116.1$& $40.8$ & $23.0$ & $62.1$ & $25.9$ & \best $22.6$ & \best $15.7$\\
      $10^6$&$15$ & $587.9$& $337.5$& $135.4$& $76.3$ & $206.8$& $85.4$ & \best $64.5$ & \best $46.0$\\\hline
\end{tabular}
\end{table}

Для каждой конфигурации ввода было создано 10 экземпляров с разными случайно сгенерированными точками. Мы измерили общее время во всех этих случаях и разделили их на 10, чтобы получить среднее время выполнения. Измерения времени выполнялись с использованием пакета Java Microbenchmark Harness с одной итерацией прогрева не менее 6 секунд, чего было достаточно для стабилизации работы программы. Был использован высокопроизводительный сервер с процессорами AMD OpteronTM 6380 и 512 GB ОЗУ, а код был запущен с виртуальной машиной OpenJDK 1.8.0 141.
Репозиторий с кодом представлен на GitHub ~\footnote{https://github.com/mbuzdalov/non-dominated-sorting/releases/tag/v0.1}, также там можно найти графики времени работы. В таблице 1 показаны только средние результаты для $ N = 5 \cdot 10 ^ 5 $ и $ 10 ^ 6 $. Видно, что гибридный алгоритм выигрывает во всех случаях, кроме $ M = 5 $ и $ M = 7 $ на гиперплоскости. Еще одно интересное наблюдение, что ENS-NDT-ONE работает быстрее, чем ENS-NDT, на экземплярах гиперкуба с $ M \leq 7 $, что означает, что предложенная эвристика действительно эффективна. Однако константа реализации ENS-NDT-ONE немного больше.

\section{Сравнение времени работы алгоритмов на худшем случае}

Так как время работы алгоритма Густавссона ENS-NDT и нового алгоритма ENS-NDT-ONE имеют квадратичную асимптотику $O(MN^2)$ в некотором худшем случаем, мы провели сравнение времени работы этих алгоритмов на таком виде данных с получившимся гибридным алгоритмом. 

Мы провели исследование на множестве точек размером до $10^5$ и разных размерностях: 3, 5, 7, 10, 15. Результаты сравнения времени работы на худшем случае для алгоритмов ENS-NDT и ENS-NDT-ONE приведены в таблице ~\ref{results_worst_case}.

\begin{table}[!ht]
\caption{Среднее время работы алгоритмов на худшем виде данных для алгоритмов ENS-NDT и ENS-NDT-ONE в секундах.}
\label{results_worst_case}
\begin{tabular}{rr|r|r|r|r}
$N$&$M$ & {Divide\&Conquer} 
        & {ENS-NDT} 
        & {ENS-NDT-ONE} 
        & {Hybrid} \\
& & {\scriptsize worst case} 
  & {\scriptsize worst case} 
  & {\scriptsize worst case} 
  & {\scriptsize worst case} \\\hline
      $10^5$&$3$  & $0.04$& $0.32$ & $0.38$ & $0.04$\\\hline
      $10^5$&$5$  & $0.05$& $4.6$ & $7.5$ & $0.7$\\\hline
      $10^5$&$7$  & $0.05$& $45.5$ & $38.3$ & $1.58$\\\hline
      $10^5$&$10$ & $0.06$& $69.1$ & $81.8$ & $3.37$\\\hline
      $10^5$&$15$ & $0.08$& $175.3$ & $185.3$ & $4.45$\\\hline
\end{tabular}
\end{table}

Эмпирический анализ времени работы показал, что данные, которые только за квадратичное время сортируются алгоритмами ENS-NDT и ENS-NDT-ONE, нашим гибридным алгоритмом сортируются несравнимо быстро. Не смотря на то, что гибридный алгоритм проигрывает по времени алгоритму на основе метода ``разделяй и властвуй'', это происходит только на специфическом, худшем виде данных, на остальных же наборах точек, как было видно в таблице ~\ref{results_worst_case}, гибридный алгоритм работает сильно быстрее алгоритма Divide\&Conquer. Это означает, что на любых видах данных наш алгоритм работает быстро.

\section{Многопоточный алгоритм недоминирующей сортировки}

Наше исследование было мотивировано очень важной практической задачей: многокритериальной задачей оптимизации управления топливом, быстрое решение которой необходимо в ходе функционирования ядерного реактора \cite{Schlunz}. Необходим быстрый алгоритм для большого количества точек до $10^6$. После успешной реализации алгоритма недоминирующей сортировки появилась идея написать параллельный алгоритм. 

Для реализации многопоточного алгоритма использовалась специфика алгоритма Буздалова, который основан на методе ``разделяй и властвуй''.  В момент рекурсивного запуска на множествах точек, ранги которых только впоследствии повлияют друг на друга, можно выполнить сортировку этих двух множеств независимо, а значит в разных потоках. Важно отметить, что в этот же момент происходит проверка переключения на новый алгоритм, то есть если по параметрам настройки гибридизации необходимо переключить алгоритм, то алгоритм Баздалова переключается на алгоритм Густавссона, иначе алгоритм Буздалова продолжает работать, но в новом потоке. 

Мы выявили две стратегиями разделения данных между потоками:
\begin{enumerate}
    \item Создавать каждый раз новые объекты всех необходимых классов.
    \item Переиспользовать ранее созданные классы, учитывая, что их может одновременно использовать несколько потоков.
\end{enumerate}

Во втором случае необходимо заранее договориться, какая часть данные, будет использоваться в какой момент, и обеспечить, чтобы пересекающиеся данные не изменялись в разных потоках в одно время. При выборе второго способа пришлось бы в самом начала выделить память на максимальный возможный размер точек. Однако, на практике нам такого количества памяти не нужно, так как в гибридном алгоритме обычно происходит переключение не на всех точках, а на частях, которые часто сильно меньше общего количества точек.

После проведения теоретического и практического сравнение затрат по памяти этих двух стратегий. Оказалось, что суммарное количество памяти, которое необходимо в первом случае примерно равно затратам по памяти во втором случае. 

Также при реализации, используя вторую стратегию, мы столкнулись с проблемой: при больших массивах размером с максимальное количество точек, алгоритм выполняется медленнее, чем аналогичный алгоритм, использующий маленькие массивы, созданные только для текущей итерации. Сравнение происходило на одном потоке. Мы пришли к выводу, что это особенности компиляции и исполнения, таким образом, остановились на первой стратегии параллелизации. 

После написания корректно работающего алгоритма мы произвели замеры времени работы алгоритма на одном, на двух и на восьми потоках. В результате был реализован гибридный параллельный алгоритм недоминирующей сортировки, который оказался быстрее однопоточной версии до $1.8$ раз на двух потоках и до $3$ раз на восьми потоках. Так как базовый алгоритм невозможно полностью приспособить к многопоточности, то есть большее время сортировка все-таки выполняется в один поток, результат считается успешным.
