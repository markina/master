\chapter{Многопоточный алгоритм недоминирующей сортировки}
\label{chapter5}

В данной главе приведено описание многопоточной версии алгоритма недоминирующей сортировки, описанного в Главе~\ref{chapter4}.

\section{Описание многопоточного алгоритма}

Наше исследование было мотивировано очень важной практической задачей: многокритериальной задачей оптимизации управления топливом, быстрое решение которой необходимо для функционирования ядерного реактора~\cite{Schlunz}. Необходим быстрый алгоритм для большого числа точек до $10^6$. После успешной реализации гибридного алгоритма недоминирующей сортировки появилась идея написать параллельный алгоритм для еще большего повышения эффективности. 

Для реализации многопоточного алгоритма использовалась специфика алгоритма Буздалова, который основан на методе <<разделяй и властвуй>>.  В момент рекурсивного запуска на множествах точек, ранги которых только впоследствии повлияют друг на друга, можно выполнить сортировку этих двух множеств независимо, а значит, в разных потоках. Важно отметить, что в этот же момент происходит проверка переключения на новый алгоритм, то есть если по параметрам настройки гибридизации необходимо переключить алгоритм, то алгоритм Буздалова переключается на алгоритм Густавссона, иначе алгоритм Буздалова продолжает работать, но в новом потоке. 

Мы выявили две стратегии разделения данных между потоками:
\begin{enumerate}
    \item Создавать каждый раз новые объекты всех необходимых классов.
    \item Переиспользовать ранее созданные классы, учитывая, что их может одновременно использовать несколько потоков.
\end{enumerate}

Во втором случае необходимо заранее договориться, какая часть данных в какой момент будет использоваться, и обеспечить гарантию того, что пересекающиеся данные не изменяются в разных потоках одновременно. При выборе второго способа пришлось бы в самом начале выделить память на максимальный возможный размер точек. Однако, на практике нам такого количества памяти не нужно, так как в гибридном алгоритме обычно происходит переключение не на всех точках, а на частях, которые часто сильно меньше общего числа точек.

После проведения теоретического и практического сравнения затрат по памяти этих двух стратегий оказалось, что суммарное количество памяти, которое необходимо в первом случае примерно равно затратам по памяти во втором случае. 

Также при реализации, используя вторую стратегию, мы столкнулись с проблемой: при больших массивах размером с максимальное число точек, алгоритм выполняется медленнее, чем аналогичный алгоритм, использующий маленькие массивы, созданные только для текущей итерации. Сравнение происходило на одном потоке. Мы пришли к выводу, что это особенности исполнения, и остановились на первой стратегии параллелизации. 

\section{Анализ времени работы многопоточного алгоритма}

После написания корректно работающего алгоритма мы произвели замеры времени работы алгоритма на одном, на двух и на восьми потоках. В результате был реализован гибридный параллельный алгоритм недоминирующей сортировки, который оказался быстрее однопоточной версии до $1.8$ раз на двух потоках и до $3$ раз на восьми потоках. Так как базовый алгоритм невозможно полностью приспособить к многопоточности, то есть большую часть времени сортировка все равно выполняется в один поток, результат считается успешным.

% TODO добавить таблицу 
