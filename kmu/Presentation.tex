\documentclass[table]{beamer}
\usetheme{CambridgeUS}

%\documentclass[handout]{beamer}
%\usetheme{Pittsburgh}
%\beamertemplatesolidbackgroundcolor{black!2}
%\setbeamertemplate{footline}[frame number]
%\usepackage{pgfpages}
%%\pgfpagesuselayout{4 on 1}[a4paper,border shrink=5mm,landscape]
%\pgfpagesuselayout{8 on 1}[a4paper,border shrink=5mm]

%%% PACKAGES
\usepackage{xcolor}
\usepackage[russian]{babel}
\usepackage[utf8]{inputenc}
\usepackage{amsmath}
\usepackage{amssymb}
\usepackage{tikz}
\usepackage{graphics}
\usepackage{appendixnumberbeamer}
%%% BEAMER SETTINGS
\setbeamertemplate{navigation symbols}{}
\setbeamertemplate{headline}{}

%%% TIKZ SETTINGS
\usetikzlibrary{fit}

%%% NEW COMMANDS
\def\pitem{\pause \item}
\newcommand{\best}{\cellcolor{gray!50!white}}

\DeclareGraphicsRule{.1}{mps}{*}{}
\DeclareGraphicsRule{.2}{mps}{*}{}
\DeclareGraphicsRule{.3}{mps}{*}{}
\DeclareGraphicsRule{.4}{mps}{*}{}
\DeclareGraphicsRule{.5}{mps}{*}{}
\DeclareGraphicsRule{.6}{mps}{*}{}

%\includeonlyframes{current} % leaves only the given frames

\title[Недоминирующая сортировка]{Методы анализа и настройки гибридных алгоритмов недоминирующей сортировки}
%\transduration{20}
\author[Маркина М., Буздалов М.]{Маркина Маргарита, Буздалов Максим}
\institute[]{Национальный исследовательский университет информационных технологий, механики и оптики}
\date{}

\begin{document}

\begin{frame}
%\transduration{20}
\begin{center}
{\scriptsize Санкт-Петербургский национальный исследовательский университет \\ информационных технологий, механики и оптики}

\vspace{1cm}

\vspace{1cm}

\vbox{\large\bfseries
Методы анализа и настройки гибридных алгоритмов недоминирующей сортировки}

\vspace{1cm}
\vspace{1cm}

{\large Маркина~М.~А., Буздалов~М.~В. \\}

\vspace{1cm}

{\large VII Конгресс молодых ученых \\}
{\large 18 апреля 2018 г. \\}

\end{center}
\end{frame}

%%%%%%%%%%%%%%%%%%%%%%%%%%%%%%%%%%%%%%%%%%%%%%%%%%%%%%%%%%%%%%%%%%%%%%%%%%%%%%%%%%%%%%%%%%%%%%%%%%%%%%%%%%%%%%%%%%%%%%%%
%\frame[label=title]{\titlepage}

\begin{frame}{Решаемая проблема}
%\transduration{20}
\begin{block}{Введение}
\begin{itemize}
\item Точка $A=(a_1,...,a_M)$ доминирует точку $B=(b_1,...,b_M)$, \\
когда $\forall ~ 1 \leq i \leq M : a_i \leq b_i$ и $\exists j : a_j < b_j $.
\item \textit{Недоминирующая сортировка} множества точек $S$ в $M$-мерном пространстве — это процедура, назначающая всем точкам из $S$ \textit{ранг}.
\item Все точки, которые не доминируются ни одной точкой из $S$, имеют ранг 0.
\item Точка имеет ранг $i+1$, если максимальный ранг среди доминирующих  её точек равен $i$.
\end{itemize}
\end{block}
\end{frame}


\begin{frame}{Решаемая проблема}
%\transduration{20}
\begin{block}{Введение}
На рисунке 3 \textbf{фронта}:

$\{a, b, c, d\}$ имеет ранг 0, $\{e, f\}$ - ранг 1, $\{g, h, i\}$ - ранг 2.
\begin{center}
\includegraphics*[height=6cm]{pic/non_dominated_sort.png}
\end{center}
\end{block}
\end{frame}

\begin{frame}{Решаемая проблема}
%\transduration{20}
\begin{block}{Актуальность}
\begin{itemize}
\item Многокритериальные эволюционные алгоритмы.
\item Алгоритм востребован в промышленных задачах.
\end{itemize}
\end{block}
\end{frame}


\begin{frame}{Решаемая проблема}
%\transduration{20}
\begin{block}{Цель исследования}
\begin{itemize}
\item Выбрать наиболее подходящие алгоритмы.
\item Приспособить их для гибридизации.
\item Разработать гибридный алгоритм.
\item Настроить параметры гибридизации.
\end{itemize}
\end{block}
\end{frame}


\begin{frame}{Решение}
%\transduration{20}
\begin{block}{Divide and Conquer + ENS-NDT}
\begin{center}
%\includegraphics*[height=3cm]{pic/illustrations.6}
\end{center}
\begin{itemize}
\item \textbf{Divide and Conquer.} M. Buzdalov, A. Shalyto. A Provably Asymptotically Fast Version of the Generalized Jensen Algorithm for Non-Dominated Sorting. (2014)
\item \textbf{ENS-NDT.} P. Gustavsson, A. Syberfeldt. A new algorithm using the non-dominated tree to improve non-dominated sorting. (2017)
\end{itemize}
\end{block}
\end{frame}

\begin{frame}{Решение}
%\transduration{20}
\begin{block}{Divide and Conquer}
\begin{itemize}
\item Разделяй и властвуй по $N$ и $M$:
\begin{itemize}
\item На каждом этапе делим на 3 множества по $k_i$ критерию текущее множество точек.
\item Если все $k_i$ в одном из подмножеств равны между собой, переходим к $k_{i-1}$.
\item Запускаемся рекурсивно.
\end{itemize}
\end{itemize}
\begin{center}
\includegraphics*[height=3cm]{pic/fast_pic.png}
\end{center}
\end{block}
\end{frame}

\begin{frame}{Решение}
%\transduration{20}
\begin{block}{ENS-NDT}
\begin{itemize}
\item Для каждого ранга поддерживается отдельное дерево.
\item Перебираем точки в лексикографическом порядке:
\begin{itemize}
\item Определяем ранг по текущему набору деревьев. 
\item Добавляем точку в соответствующее рангу дерево. 
\end{itemize}
\end{itemize}
\begin{center}
\includegraphics*[height=3.5cm]{pic/ndtree_original.png}
\end{center}
\end{block}
\end{frame}

\begin{frame}{Решение}
%\transduration{20}
\begin{block}{ENS-NDT-ONE}
\begin{itemize}
\item Адаптация алгоритма ENS-NDT
\begin{itemize}
\item Структура будет состоять из единственного дерева.
\item В узлах дерева будет содержаться максимальный ранг на поддереве.
\begin{itemize}
  \item Оптимизация на этапе определения ранга точки.
\end{itemize}
\end{itemize}
\end{itemize}
\begin{center}
\includegraphics*[height=3.5cm]{pic/ndtree_new.png}
\end{center}
\end{block}
\end{frame}

\begin{frame}{Решение}
%\transduration{20}
\begin{block}{Асимптотика}
\begin{itemize}
\item Divide and Conquer $O(N \log^{M-1}N)$.
\item ENS-NDT
\begin{itemize}
\item $O(N^{1.43})$ для случано сгенерированых точек в гиперкубе.
\item В худшем случае $O(MN^{2})$.
\end{itemize}
\item ENS-NDT-ONE
\begin{itemize}
\item $O(MN^{1.58})$ для случано сгенерированых точек в гиперкубе.
\item В худшем случае $O(MN^{2})$.
\end{itemize}
\end{itemize}
\end{block}
\end{frame}

\begin{frame}{Решение}
%\transduration{20}
\begin{block}{Гибридизация}
  \begin{itemize}
  \item Запускаем алгоритм Divide and Conquer, согласно некоторыем правилам переключаемся на алгоритм ENS-NDT-ONE.
  \item Моменты смены алгоритма:
    \begin{itemize}
    \item HelperA
      \begin{itemize}
      \item Входные данные: множество точек $S$ с предварительными рангами.
      \item Результат выполнения: множество точек $S$ с обновленными рангами.
      \end{itemize}
    \item HelperB  
      \begin{itemize}
      \item Входные данные: множество точек $L$ с окончательными рангами и $R$ с предварительными рангами.
      \item Результат выполнения: множество точек $R$ с обновленными рангами по множеству $L$.
      \end{itemize}
    \end{itemize}
  \end{itemize}
\end{block}
\end{frame}

\begin{frame}{Решение}
%\transduration{20}
\begin{block}{Настройка параметров гибридного алгоритма}
\begin{itemize}
\item Параметры гибридного алгоритма определяют максимальные размеры множеств точек для каждой размерности, при котором алгоритм Divide and Conquer переключается на алгоритм ENS-NDT-ONE.
\item Параметры основаны на экспериментальных исследованиях скорости работы и являются константами.
\end{itemize}
\end{block}
\end{frame}

\begin{frame}{Решение}
%\transduration{20}
\begin{block}{Обоснование эффективности гибридного алгоритма}
\begin{itemize}
\item Экспериментально получено, что для эффективной работы гибридного алгоритма размер точек в момент переключения на алгоритм ENS-NDT-ONE не должен привышать $10^4 - 2 \cdot 10^4$.
\item Таким образом, худший случай для алгоритма ENS-NDT-ONE с асимптотикой $O(MN^2)$ незначительно влияет на асимптотику гибридного алгоритма.
\end{itemize}
\end{block}
\end{frame}

\begin{frame}{Результат}
%\transduration{20}
\begin{block}{Сравнение скоростей}

\begin{center}
\begin{table}[!ht]
\begin{tabular}{rr|rr|rr|rr|rr}
$N$&$M$ & \multicolumn{2}{c|}{D\&C} 
        & \multicolumn{2}{c|}{ENS-NDT} 
        & \multicolumn{2}{c|}{ENS-NDT-ONE} 
        & \multicolumn{2}{c}{Hybrid} \\
& & {\scriptsize cube} & {\scriptsize plane} 
  & {\scriptsize cube} & {\scriptsize plane} 
  & {\scriptsize cube} & {\scriptsize plane} 
  & {\scriptsize cube} & {\scriptsize plane} \\\hline  % hyper
      $10^6$&$3$  & $2.82$ & $1.60$ & $5.25$ & $1.61$ & $4.25$ & $1.65$ & \best $2.63$ & \best$1.50$\\\hline
      $10^6$&$5$  & $45.2$ & $33.0$ & $26.3$ & \best$5.22$ & $18.2$ & $5.82$ & \best $17.2$ & $12.8$\\\hline
      $10^6$&$7$  & $191.5$& $120.2$& $55.4$ & $19.4$ & $46.1$ & \best$18.9$ & \best $26.8$ & $20.1$\\\hline
      $10^6$&$10$ & $478.8$& $228.6$& $84.8$ & $48.1$ & $104.8$& $55.0$ & \best $41.0$ & \best $33.0$\\\hline
      $10^6$&$15$ & $587.9$& $337.5$& $135.4$& $76.3$ & $206.8$& $85.4$ & \best $64.5$ & \best $46.0$\\\hline
\end{tabular}
\end{table}
\end{center}
\end{block}
\end{frame}

\begin{frame}{Результат}
%\transduration{20}
\begin{block}{Выводы}
\begin{itemize}
\item Гибридный алгоритм быстрее обоих родительских алгоритмов.
\item Нам неизвестны публикации результатов сортировки множеств точек размером $10^6$ большой размерности с приемлемым временем выполнения.
\item Алгоритм адаптирован для многопоточного выполнения, ускорение составляет до $1.8$ на двух потоках и до трех раз на восьми потоках.
\item По результатом этой работы была подготовлена для публикации статья на конференцию на PPSN $2018$, и мы ждем рецензий. 
\end{itemize}
\end{block}
\end{frame}

\begin{frame}{}
\begin{center}
Спасибо за внимание!
\end{center}
\end{frame}

\appendix

%\begin{frame}{Дополнительные материалы}
%
%\end{frame}

\end{document}